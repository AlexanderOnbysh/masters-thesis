\chapter{Research Hypothesis and Problem}
\label{Hypothesis}

\section{Hypotheses}
\paragraph{Model Comparison hypothesis:} This project's main objective is to create a model for human pose estimation based on point cloud using a capsule-based neural network, which shows competitive performance on well-known benchmarks.

\paragraph{Noise resistance hypothesis:} The impact of noise in the training dataset on the capsule-based model should be less compared to non-capsule-based models. A hypothesis is made based on 2D image recognition using capsule networks \cite{sabour_dynamic_2017,}.

\paragraph{Dataset size hypothesis:} The dataset's size for compatible results should be smaller for capsule-based networks compared to non-capsule ones. The assumption is made based on experiments presented is \cite{sabour_dynamic_2017,wang_capsule_2020,gritsevskiy_capsule_2018} based on 2D image classification.

\section{Problems}
\paragraph{Noise problem.} To achieve the project goal mentioned above, we need to generate realistic noise for point cloud data. We need to compare how noisy data influence capsule-based and not capsule-based models. 

\paragraph{Models' retrain problem.} To achieve the project's third goal, we need to retrain the reference SOTA \footnote{stage of the art} model with truncated training data. After this compare retrained model with the reference capsule-based model.